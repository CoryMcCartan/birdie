\documentclass[11pt]{article}

%% === margins ===
\addtolength{\hoffset}{-0.8in} \addtolength{\voffset}{-0.8in}
\addtolength{\textwidth}{1.6in} \addtolength{\textheight}{1.6in}

%% JASA format with 12pt, spacingset = 1.83
%\addtolength{\hoffset}{-0.3in} \addtolength{\voffset}{-1.2in}
%\addtolength{\textwidth}{.6in} \addtolength{\textheight}{2.1in}
%\pdfminorversion=4


%% === basic packages ===
\usepackage{latexsym,multirow}
\usepackage{amssymb,amsmath,bm,mathtools,physics}
\usepackage{cancel}
\usepackage{graphicx}
\usepackage{booktabs}

\usepackage{enumerate}

%% === tikz ===
\usepackage{tikz}
\usetikzlibrary{bayesnet}

%% === bibliography packages ===
\usepackage{natbib}
\bibliographystyle{pa}
%% === hyperref options ===
% \usepackage{color}
\usepackage[pdftex, bookmarksopen=true, bookmarksnumbered=true,
pdfstartview=FitH, breaklinks=true, urlbordercolor={0 1 0}, citebordercolor={0 0 1}]{hyperref}
\usepackage{colortbl}
\usepackage{subcaption}

% === dcolumn package ===
\usepackage{dcolumn}
\newcolumntype{.}{D{.}{.}{-1}}
\newcolumntype{d}[1]{D{.}{.}{#1}}
% === theorem package ===
\usepackage{theorem}
\theoremstyle{plain}
\theoremheaderfont{\scshape}
\newtheorem{assumption}{Assumption}
\newtheorem{example}{Example}
\newtheorem{definition}{Definition}
\newtheorem{corollary}{Corollary}
\newtheorem{property}{Property}
\newtheorem{proposition}{Proposition}
\newtheorem{theorem}{Theorem}
\newtheorem{remark}{Remark}
\newtheorem{defi}{Definition}
\newtheorem{thm}{Theorem}
\newtheorem{lemma}{Lemma}

% === some special symbols
\newcommand{\qed}{\hfill \ensuremath{\Box}}
\newcommand{\iid}{\mathrel{\stackrel{iid}{\sim}}}
\newcommand{\indep}{\stackrel{indep.}{\sim}}
\newcommand{\ind}{\mbox{$\perp\!\!\!\perp$}}
\newcommand{\nind}{\mbox{$\not\perp\!\!\!\perp$}}
\def\independenT#1#2{\mathrel{\rlap{$#1#2$}\mkern2mu{#1#2}}}
\DeclareMathOperator{\sgn}{sgn}
\DeclareMathOperator{\diag}{diag}
\providecommand{\norm}[1]{\lVert#1\rVert}
\newcommand{\argmax}{\operatornamewithlimits{argmax}}
\newcommand{\argmin}{\operatornamewithlimits{argmin}}

% ==== rotating package ===
\usepackage{rotating}

% ==== dotted lines in tables ===
%\usepackage{arydshln}

% == spacing between sections and subsections
\usepackage[compact]{titlesec}
%\usepackage{times}

\allowdisplaybreaks

\newcommand\spacingset[1]{\renewcommand{\baselinestretch}%
  {#1}\small\normalsize}

\let\S\relax % to avoid spurious warnings
\DeclareRobustCommand{\S}{%
  \ifmmode
    \mathsection
  \else
    \textsection~%
  \fi
}
% === versions control ===
\usepackage{versions}
%\excludeversion{real}
%\includeversion{synth}
\excludeversion{synth}
\includeversion{real}
\graphicspath{{figs/}}


%%%%%%%%%%%%%%%%%%%%%%%%%%%%%%%%%%%%%%%%%%%%%%%%%%%%%%%%%%%%%%%%%%%%%%

%% === submission
\newcommand{\blind}{0}

\def\letas{\mathrel{\mathop{=}\limits^{\triangle}}}

\newcommand*{\QEDB}{\hfill\ensuremath{\square}}

\newcommand{\bone}{\mathbf{1}}
\newcommand{\red}{\color{red}}
\newcommand{\blue}{\color{blue}}
\newcommand{\Beta}{\text{Beta}}
\newcommand{\Binomial}{\text{Binomial}}
\newcommand{\Bern}{\text{Bern}}
\newcommand{\Expo}{\text{Expo}}
\newcommand{\Pois}{\text{Pois}}
\newcommand{\Unif}{\text{Unif}}
\newcommand{\Gammad}{\text{Gamma}}
\newcommand{\Dir}{\mathrm{Dir}}
\newcommand{\Cat}{\mathrm{Cat}}
\newcommand{\logit}{\text{logit}}
\newcommand{\expit}{\text{expit}}
\newcommand{\mexpit}{\text{mexpit}}

\newcommand{\pr}{\text{pr}}
\renewcommand{\var}{\textnormal{var}}
\newcommand{\cov}{\textnormal{cov}}
\newcommand{\sumN}{\sum_{i=1}^N}

\newcommand{\bZ}{\bm{Z}}
\newcommand{\bD}{\bm{D}}
\newcommand{\bz}{\bm{z}}
\newcommand{\bd}{\bm{d}}
\newcommand{\oD}{\overline{D}}
\newcommand{\oY}{\overline{Y}}
\newcommand{\wD}{\widehat{D}}
\newcommand{\wY}{\widehat{Y}}

\newcommand{\wtphi}{\widetilde{\bm{\phi}}}

\newcommand{\bW}{\bm{W}}
\newcommand{\bw}{\bm{w}}
\newcommand{\cZ}{\mathcal{Z}}
\newcommand{\bS}{\bm{S}}
\newcommand{\cS}{\mathcal{S}}
\newcommand{\cR}{\mathcal{R}}
\newcommand{\cG}{\mathcal{G}}
\newcommand{\bs}{\bm{s}}
\newcommand{\bY}{\bm{Y}}
\newcommand{\E}{\mathbb{E}}
\newcommand{\bx}{\mathbf{x}}
\newcommand{\bX}{\mathbf{X}}
\newcommand{\cX}{\mathcal{X}}
\newcommand{\cY}{\mathcal{Y}}
\newcommand{\bI}{\mathbf{I}}
\newcommand{\bM}{\mathbf{M}}
\newcommand{\bP}{\mathbf{P}}
\newcommand{\bQ}{\mathbf{Q}}
\newcommand{\bV}{\mathbf{V}}
\newcommand{\bU}{\mathbf{U}}

\newcommand{\ATE}{\textsf{ATE}}
\newcommand{\bepsilon}{\bm{\epsilon}}
\newcommand{\bbeta}{\bm{\beta}}
\newcommand{\balpha}{\bm{\alpha}}
\newcommand{\bphi}{\bm{\phi}}
\newcommand{\btheta}{\bm{\theta}}
\newcommand{\bgamma}{\bm{\gamma}}

\begin{document}

\newcommand{\tit}{Unbiased Estimation of Racial Disparity when Race is
Not Observed}
%
%%%%%%%%%%%%%%%%%%%%%%%%%%%%%%%%%%%%%%%%%%%%%%%%%%%%%%%%%%%%%%%%%%%%%%%%%

\spacingset{1.25}

\if0\blind

{\title{{\bf\tit}\thanks{}}

  \author{}


  \date{\today
}

\maketitle

}\fi


\if1\blind
\title{\bf \tit}


\maketitle
\fi

\pdfbookmark[1]{Title Page}{Title Page}

\thispagestyle{empty}
\setcounter{page}{0}

\begin{abstract}


 \medskip
 \noindent {\bf Keywords:}

\end{abstract}


%%%%%%%%%%%%%%%%%%%%%%%%%%%%%%%%%%%%%%%%%%%%%%%%%%%

\clearpage
%\spacingset{1.83}
\spacingset{1.5}

\section{Introduction}

\section{The Proposed Methodology}

\begin{figure}[t]
  \begin{center}
    \tikzstyle{main node}=[circle,draw,font=\sffamily\Large\bfseries]
    \tikzstyle{sub node}=[circle,draw,dashed,font=\sffamily\Large\bfseries]
    \begin{tikzpicture}[->,>=stealth',shorten >=1pt,auto,node distance=3cm,thick]

      \node[sub node] (1) {$R$};
      \node[main node] (7) [left of=1] {$S$};
      \node[main node] (2) [below of=1] {$G$};
      \node[main node] (4) [right of=1] {$X$};
      \node[main node] (3) [below of=4] {$Z$};
      \node[main node] (5) [right of=4] {$Y$};
      \node[sub node] (6) [below right =1.5cm and .8cm of 2] {$U$};

      \path[every node/.style={font=\sffamily\small}]
      (1) edge node  {} (2)
      (1) edge node  {} (3)
      (1) edge node  {} (4)
      (1) edge node  {} (7)
      (2) edge node  {} (3)
      (2) edge node  {} (4)
      (3) edge node  {} (4)
      (3) edge node  {} (5)
      (4) edge node  {} (5)
      (6) edge node  {} (2)
      (6) edge node  {} (3)
      (6) edge node  {} (4);
    \end{tikzpicture}
  \end{center}
  \caption{DAG}
\end{figure}

\begin{itemize}
\item $R_i \in \cR= \{1,2,\ldots,K\}$: race
\item $Y_i \in \cY = \{1, 2, \ldots, L_Y\}$: outcome variable (categorical)
\item $Z_i \in \cZ$: $K_{Z}$-dimensional vector of input variables for which
  the population data are available (e.g., income, family structure)
\item $X_i \in \cX$: $K_{X}$-dimensional vector of input variables for which
  the population data are not available
\item $S_i \in \cS = \{1, 2, \ldots, L_S\}$: surnames
\item $G_i \in \cG = \{1, 2, \dots, L_G\}$: residence geolocation
\item $U_i$: unobserved confounders
\end{itemize}

\subsection{Identification}

\begin{align*}
  & \Pr(Y_i = y, G_i = g, S_i = s, X_i = x, Z_i = z) \\
  \ = \ &  \sum_{r \in \cR} \Pr(Y_i = y, G_i = g, S_i = s, X_i = x, Z_i = Z\mid
          R_i = r) \Pr(R_i = r) \\
  \ = \ &  \sum_{r \in \cR} \Pr(Y_i = y, G_i = g, Z_i = z, X_i = x \mid
          R_i = r) \Pr(S_i = s \mid R_i = r)\Pr(R_i = r) \\
  \ = \ &  \sum_{r \in \cR} \Pr(Y_i = y \mid Z_i = z, X_i = x) \Pr(G_i
          = g, Z_i = z, X_i = x \mid R_i = r)\Pr(S_i = s \mid R_i = r)\Pr(R_i = r) \\
  \ = \ &  \Pr(Y_i = y \mid Z_i = z, X_i = x) \sum_{r \in \cR} \Pr(X_i
           = x \mid G_i = g, Z_i = z, R_i = r)\pi_r(g,s,z)
\end{align*}
where
$\pi_r(g,s,z) = \Pr(G_i = g, Z_i = z \mid R_i = r)\Pr(S_i = s \mid R_i
= r)\Pr(R_i = r)$.  This implies,
\begin{equation*}
  \Pr(G_i = g, S_i = s, X_i = x, Z_i = z) \ = \ \sum_{r \in \cR} \Pr(X_i
  = x \mid G_i = g, Z_i = z, R_i = r)\pi_r(g,s,z)
\end{equation*}
Define
\begin{align}
  \bbeta(g,x,z) \ = & \ \begin{pmatrix}
    \Pr(X_i = x \mid G_i = g, Z_i = z, R_i = 1) \\
    \Pr(X_i = x \mid G_i = g, Z_i = z, R_i = 2) \\
    \vdots \\
    \Pr(X_i = x \mid G_i = g, Z_i = z, R_i = K)
  \end{pmatrix},
  \\
  \Pi(g,z) \ =  & \ \begin{pmatrix}
    \pi_1(g,1,z) & \pi_2(g,1,z) & \cdots & \pi_K(g,1,z) \\
    \pi_1(g,2,z) & \pi_2(g,2,z) & \cdots & \pi_K(g,2,z) \\
    \vdots & \vdots & \ddots & \vdots \\
    \pi_1(g,L_S,z) & \pi_2(g,L_S,z) & \cdots & \pi_K(g,L_S,z) \\
  \end{pmatrix},
  \\
  \Lambda(g,x,z) \ = & \ \begin{pmatrix}
    \Pr(G_i = g, S_i = 1, X_i = x, Z_i = z) \\
    \Pr(G_i = g, S_i = 2, X_i = x, Z_i = z) \\
    \vdots \\
   \Pr(G_i = g, S_i = L_S, X_i = x, Z_i = z) ) \\
  \end{pmatrix}
\end{align}
Then, the identification equation is given by,
\begin{equation}
  \Lambda(g,x,z) \ = \ \Pi(g,z) \bbeta(g,x,z)
\end{equation}
The least squares estimator is
\begin{equation}
  \widehat{\bbeta(g,x,z)} \ = \ \left(\Pi(g,z)^\top\Pi(g,z)\right)^{-1} \Pi(g,z)^\top \widehat{\Lambda(g,x,z)}
\end{equation}
where $\widehat{\Lambda(g,x,z)}$ is the corresponding sample
proportion. This is identifiable so long as $L_S > K$.  The racial
disparity estimate is given by,
\begin{equation}
  \Pr(Y_i = y \mid R_i = r) \ = \ \sum_{z \in \cZ} \sum_{x \in \cX}
  \sum_{g \in \cG} \Pr(Y_i = y \mid X_i = x, Z_i = z)
  \widehat{\bbeta_r(g,x,z)} \Pr(G_i = g, Z_i = z \mid R_i = r)
\end{equation}

\subsection{Bayesian model}

\begin{align}
  X_i \mid Z_i = z, G_i = g,  R_i = r \ &\sim \ \Cat(\vb p_{X|r}(g,z))  \\
  Z_i, G_i \mid R_i  = r \ &\sim \ \Cat(\vb p_{ZG|r}) \\
  S_i \mid R_i = r \ &\sim \ \Cat(\vb p_{S|r})  \\
  R_i \ &\iid \ \Cat(\vb p_r) \\
  \vb p_{X|r}(g,z) \ &\iid \ \Dir(\bm\alpha_{X|r})
\end{align}
where $\vb p_{ZG|r}, \vb p_{S|r}, \vb p_r$ are obtained from the census and other
data sets.  The joint posterior
\begin{align}
  & \Pr(R, \vb p_{X|r} \mid X, Z, G, S) \\
  \propto \ & \prod_{i=1}^N \Pr(X_i \mid Z_i, G_i,
  R_i, \vb p_{X|R_i}(G_i, Z_i))\Pr(Z_i, G_i \mid R_i) \Pr(S_i \mid R_i) \Pr(R_i) \prod_{r
              \in \cR} P(\vb p_{X|r}) \\
  \propto \ & \prod_{r \in \cR} \prod_{i=1}^N  \left[p_r \prod_{s \in \cS}
             p_{s|r}^{\mathbf{1}\{S_i = s\}} \prod_{g \in \cG} \prod_{z \in \cZ}
             \left[ \left(\prod_{x \in \cX} p_{x|r}(g,z)^{\mathbf{1}\{X_i = x\}}\right)
              p_{zg|r}\right]^{\mathbf{1}\{G_i = g, Z_i =
              w\}} \right]^{\mathbf{1}\{R_i = r\}} \\
  & \quad \times \prod_{r \in \cR} \prod_{x \in \cX} \prod_{g \in \cG}
    \prod_{z \in \cZ} p_{x|r}(g,z)^{\alpha_{xr}-1} \\
  = \ &  \prod_{r \in \cR} \prod_{x \in \cX} \prod_{g \in \cG}
    \prod_{z \in \cZ} p_{x|r}(g,z)^{\sum_{i=1}^N \mathbf{1}\{G_i = g,
        Z_i = z, X_i = x, R_i = r\} + \alpha_{xr}-1} \times \prod_{r
        \in \cR} p_r^{\sum_{i=1}^N \mathbf{1}\{R_i = r\}} \\
  & \times \prod_{r \in \cR} \prod_{g \in \cG}
    \prod_{z \in \cZ}p_{zg|r}^{\sum_{i=1}^N \mathbf{1}\{G_i = g,
        Z_i = z, R_i = r\}} \times \prod_{r  \in \cR} \prod_{s \in \cS}
    p_{s|r}^{\sum_{i=1}^N \mathbf{1}\{S_i = s, R_i = r\}}
\end{align}
Integrating out $\vb p_{X|r}$ gives
\begin{align}
  &  \Pr(R \mid X, Z, G, S) \\
\propto  & \prod_{r \in \cR} \prod_{g \in \cG}
    \prod_{z \in \cZ} \frac{\prod_{x \in \cX}\Gamma(\sum_{i=1}^N \mathbf{1}\{G_i =
           g, X_i = x,  Z_i = z, R_i = r\} + \alpha_{xr})}{\Gamma(\sum_{i=1}^N
           \mathbf{1}\{G_i = g,
           Z_i = z, R_i = r\} + \sum_{x\in\cX} \alpha_{xr})} \times \prod_{r
        \in \cR} p_r^{\sum_{i=1}^n \mathbf{1}\{R_i = r\}} \\
  & \times \prod_{r \in \cR} \prod_{g \in \cG}
    \prod_{z \in \cZ}p_{zg|r}^{\sum_{i=1}^N \mathbf{1}\{G_i = g,
        Z_i = z, R_i = r\}} \times \prod_{r  \in \cR} \prod_{s \in \cS}
    \vb p_{s|r}^{\sum_{i=1}^N \mathbf{1}\{S_i = s, R_i = r\}}
\end{align}
Using the fact that $\Gamma(x+y)=x^y\Gamma(x)$ for $y \in \{0,1\}$, we
have,
\begin{align}
  & \Pr(R_i = r \mid R_{-i}, G_i = g, S_i = s, X_i = x, Z_i = z) \\
  \propto \ &
              \frac{n_r^{-i}(g,x,z)+\alpha_{xr}}{m_r^{-i}(g,z)+\sum_{x
              \in \cX}\alpha_{xr}} \times p_r \times p_{zg|r}
              \times p_{s|r}
\end{align}
where
$n_r^{-i}(g,x,z) = \sum_{i^\prime\ne i} \mathbf{1}\{G_{i^\prime} = g,
X_{i^\prime} = x, Z_{i^\prime} = z, R_{i^\prime} = r\}$ and
$m_r^{-i}(g,z)=\sum_{i^\prime\ne i} \mathbf{1}\{G_{i^\prime} = g,
Z_{i^\prime} = z, R_{i^\prime} = r\}$.  We can use the Gibbs sampler
to approximate the posterior by keeping track of $n_r^{-i}(g,z)$ and
$m_r^{-i}(g,z)$.

\subsection{Calibration}

\begin{equation}
  Y_i \ \ind \ \pi(G_i, S_i, w) \mid X_i = x, Z_i = z
\end{equation}
For each $x$ and $z$, we use
\begin{equation}
\hat{\omega} \ = \  \argmin_\omega \mathcal{D}(\omega, \omega^\ast)
\end{equation}
subject to
\begin{align}
  \omega \ \ge \ 0, \quad \sum_{r=1}^K \omega_{ir} \ = \ 1 & \\
  \frac{1}{\sum_{i: X_i = x, Z_i = z} \mathbf{1}\{Y_i = y\}} \sum_{i:
  X_i  = x, Z_i = z}
  \mathbf{1}\{Y_i = y\} \omega \ = \ & \frac{1}{n_{xw}} \sum_{i: X_i = x, Z_i = z} \mathbf{1}\{Y_i = y\}
\end{align}
for each $y$ where $\mathcal{D}$ is a suitable distance measure and
$\omega^\ast$ is the posterior estimate of
$\Pr(R_i \mid G_i, S_i, X_i, Z_i)$.

\pdfbookmark[1]{References}{References}
\spacingset{1.5}
\bibliography{my,imai}

\end{document}

